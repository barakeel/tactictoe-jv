\documentclass[]{scrartcl}
\usepackage{csquotes}
\MakeOuterQuote{"}

%opening
\title{Response for JAR submission}
\subtitle{TacticToe: Learning to Prove with Tactics}
\author{\mbox{Thibault Gauthier} \and \mbox{Cezary Kaliszyk} \and \mbox{Josef
		Urban} \and \mbox{Ramana Kumar} \and \mbox{Michael Norrish}}

\usepackage{blindtext} 
\usepackage{framed}
%\usepackage[parfill]{parskip}

\setlength{\parskip}{1.5mm}
\setlength{\parindent}{0mm}

\begin{document}

\maketitle

We thank the reviewers for the valuable comments and questions. We have 
attempted to correct all the issues suggested in the reviews and we give 
further answers to the questions raised in this response letter.

\section*{Reviewer 1}

\begin{leftbar}
No data is provided to help other researchers reproduce this experiment.  I 
consider this is a flaw of the paper.  I think the paper should be improved in 
the direction of making the experiment reproducible by other researchers, as 
much as possible. The paper provides numerical evaluation of its results, 
mostly in the form of 
percentage of success under a given timeout. I would also be interesting to 
know how much it takes to re-run the whole process of learning and then using 
for the complete database.
\end{leftbar}

A paragraph on ``Reproducibility'' was added to the Subsection ``Methodology''. 
There, 
we talk about the 
reproducibility of our results with information about the availability of our 
source code (part of the HOL4 standard distribution) and the 
time it takes to reproduce the results (learning and evaluation). 

\section*{Reviewer 2}

\begin{leftbar}
The only reservation I have with the paper is that it is not clear to me 
whether and how the proof tool is able to generate two commonly-used tactics. 
Firstly, generating witnesses to existential goals using \texttt{EXISTS\_TAC 
tm}. If these 
tactics are out of scope for the proof tool, then are all goals containing 
existential
quantifiers impossible for the tool to prove? Secondly, reducing a goal using 
an implication theorem using \texttt{MATCH\_MP\_TAC thm}. Often such 
implication theorems 
are
highly specific to a single theory or even theorem, so how can tactics like 
this be effectively learned? In my opinion the paper would be improved by a 
brief
discussion on how these tactics could be learned, either in the present or a 
future system.
\end{leftbar}

The current version of TacticToe cannot synthetize non-obvious 
witness, unless it leverages some clever 
tactics to perform this task. In practice, the number of theorems containing 
an existential quantifiers is significant (about 10 percent) but they are 
specialized tactics that can solve problems containing existentials (e.g. 
linear arithmetic). It is true that as TacticToe improves this weakness 
may become more palatable and methods coming from proof planning, inductive 
theorem proving or even machine learning in general, could be integrated.
Possible solutions of the problems of predicting exact arguments are now 
proposed in Section ``Abstraction''. And we give examples for the two most 
common type of arguments appearing in HOL4 tactics (terms and theorems).

\section*{Reviewer 3}

\begin{leftbar}
The TacticToe system seems to particularly shine on inductive proofs over
recursive data structures, as the authors themselves note. It seems
therefore that the main place it wins over standard automated provers is
that it can recognize when and how to perform proof by induction. This
suggests that it may be less effective in other domains (e.g. where one has
to synthesize non-obvious witness terms) but also suggests to me that the
authors should also be comparing with the extensive literature on the
automation of inductive proof, e.g. Boyer-Moore, Bundy and "Proof
Planning", "inductionless induction" in rewriting, and the concrete
realizations in HOL family theorem provers by Boulton and Papapanagiotou.

The authors might also say a bit more about how much of this is specific to
HOL4, and how much would change if one wanted to implement something
similar in another prover either with a similar (HOL Light, Isabelle/HOL)
or different (ACL2, Coq, PVS) logic. While the authors give quite an
explicit discussion of the method, I did not see a link to the actual code.
Is this available?
\end{leftbar}

A paragraph on Proof Planning was added in the Section ``Related Work'' 
comparing aspects of proof planning to our work: proof automation, tactic 
generalization and tactic synthesis. Term synthesis limitations are
addressed in our reply to Reviewer 2.
The TacticToe framework is mostly dependent on the proof script language of the 
prover and not its logical foundation. Indeed, tactics can be seen as an 
interface between the user and the logical foundations. We now added a 
paragraph on ``Portability'' in the Subsection ``Methodology'', where we 
discuss how to port our work to a different ITP. The source code is part of the 
standard HOL4 distribution and a link is provided in the same subsection.
\end{document}
