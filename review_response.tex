\documentclass[]{scrartcl}
\usepackage{csquotes}
\MakeOuterQuote{"}

%opening
\title{Response for JAR submission}
\subtitle{TacticToe: Learning to Prove with Tactics}
\author{\mbox{Thibault Gauthier} \and \mbox{Cezary Kaliszyk} \and \mbox{Josef
		Urban} \and \mbox{Ramana Kumar} \and \mbox{Michael Norrish}}

\usepackage{blindtext} 
\usepackage{framed}
%\usepackage[parfill]{parskip}

\setlength{\parskip}{1.5mm}
\setlength{\parindent}{0mm}

\begin{document}

\maketitle

We thank the reviewers for the valuable comments and questions. We have 
attempted to correct all the issues suggested in the reviews and we give 
further answers to the questions raised in this response letter.

\section*{Reviewer 1}

\begin{leftbar}
No data is provided to help other researchers reproduce this experiment.  I 
consider this is a flaw of the paper.  I think the paper should be improved in 
the direction of making the experiment reproducible by other researchers, as 
much as possible. The paper provides numerical evaluation of its results, 
mostly in the form of 
percentage of success under a given timeout.   I would also be interesting to 
know how much it takes to re-run the whole process of learning and then using 
for the complete database.
\end{leftbar}

\section*{Reviewer 2}

\begin{leftbar}
The only reservation I have with the paper is that it is not clear to me 
whether and how the proof tool is able to generate two commonly-used tactics. 
Firstly,
generating witnesses to existential goals using "EXISTS_TAC tm". If these 
tactics are out of scope for the proof tool, then are all goals containing 
existential
quantifiers impossible for the tool to prove? Secondly, reducing a goal using 
an implication theorem using "MATCH_MP_TAC th". Often such implication theorems 
are
highly specific to a single theory or even theorem, so how can tactics like 
this be effectively learned? In my opinion the paper would be improved by a 
brief
discussion on how these tactics could be learned, either in the present or a 
future system.
\end{leftbar}

A paragraph about predicting arrguments for these commonly used tactics has 
been added. more later

\section*{Reviewer 3}

\begin{leftbar}
The TacticToe system seems to particularly shine on inductive proofs over
recursive data structures, as the authors themselves note. It seems
therefore that the main place it wins over standard automated provers is
that it can recognize when and how to perform proof by induction. This
suggests that it may be less effective in other domains (e.g. where one has
to synthesize non-obvious witness terms) but also suggests to me that the
authors should also be comparing with the extensive literature on the
automation of inductive proof, e.g. Boyer-Moore, Bundy and "Proof
Planning", "inductionless induction" in rewriting, and the concrete
realizations in HOL family theorem provers by Boulton and Papapanagiotou.

The authors might also say a bit more about how much of this is specific to
HOL4, and how much would change if one wanted to implement something
similar in another prover either with a similar (HOL Light, Isabelle/HOL)
or different (ACL2, Coq, PVS) logic. While the authors give quite an
explicit discussion of the method, I did not see a link to the actual code.
Is this available?
\end{leftbar}

More details for the re



\end{document}
